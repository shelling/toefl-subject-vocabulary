

\documentclass[twoside,b5paper]{book}

\usepackage{color}              % Text Color 
\usepackage{url}
\usepackage{fontspec}
\usepackage{graphicx}
\usepackage{textcomp}
\usepackage{tipa}               % TeX International Phonetic Alphabet

\setmainfont{BiauKai}
\newfontfamily\kai{BiauKai}
\newfontfamily\arial{Arial}

\evensidemargin=0pt
\oddsidemargin=0pt

\parindent=0pt
\marginparwidth=60pt
\topmargin=0pt
\headheight=0pt
\headsep=50pt

\voffset=0pt
\hoffset=0pt

\XeTeXlinebreaklocale "zh"

\begin{document}
  \title{TOEFL Subject Vocabulary\\ 托福學科單字}
  \author{馬翎 編繹 shelling 整理}
  \maketitle

  \tableofcontents % uncomment this line to make toc

  \chapter{Astronomy天文學}
  \begin{enumerate}
    \item
      galaxy \textipa{/"g\ae{}l@ksI/} 銀河系\\
      elliptical galaxy 橢圓的星系\\
      spiral galaxy 螺旋星系
    \item
      nebula \textipa{/"nEbj@l@/} 星雲\\
      spiral nebula 漩渦星座\\
      nebular 星雲的\\
      constellation \textipa{/""kAnst@"leS@n/} 星座 \textcolor{red}{*}\marginpar{\scriptsize Constellation is a group of stars visible from Earth that forms a distinctive pattern and has a name}\\
      star cluster 星團
    \item
      radiation 輻射線; 放射 (v.$\to$ radiate)
    \item
      magnetic field 磁場\\
      magnetic storm 磁暴\\
      magnetic pole 磁極\\
      magnetosphere (圍繞地球的)磁力層\footnote{地球大氣層外的外圈,厚數千英里,受地球磁場控制}\\
      magnitude (星的)光度
    \item
      particles; dust 微塵; 微粒
    \item
      interstellar \textipa{/""Int\textrhookschwa"stEl\textrhookschwa/} 位於各星之間的; 星際的\\
      stellar \textipa{/"stEl\textrhookschwa/} 星的\\
      stellar luminosity \textipa{/""lum@"nAs@tI/} 星的光度\\
      luminous \textipa{/"lum@n@s/} 發光的
    \item
      the solar system 太陽系\\
      solar spots / sunspots 太陽黑點\\
      Mercury 水星\\
      Venus 金星\\
      Earth 地球\\
      Mars 火星\\
      Jupiter 木星\textcolor{red}{*}\marginpar{\scriptsize In the solar system, Jupiter is the largest planet.}\\
      Saturn \textipa{/"s\ae{}t\textrhookschwa{}n/} 土星\\
      Uranus \textipa{/ju"ren@s/} 天王星\\
      Neptune \textipa{/""nEptjun/} 海王星\\
      Pluto \textipa{/"pluto/} 冥王星
    \item
      sphere \textipa{/sfIr/} 球狀體\\
      hemisphere \textipa{/"hEm@s""fIr/} 半球 \textcolor{red}{*}\marginpar{\scriptsize the Northern / Southern / Eastern / Wesern Hemisphere 北/南/東/西半球}\\
    \item
      celestial \textipa{/s@"lEstS@l/} 天上的\\
      celestial body 天體
    \item
      axis \textipa{/"\ae{}ksIs/} 軸 \textcolor{red}{*}\marginpar{\scriptsize The Earth revolves around the sun.}\marginpar{\scriptsize The Earth spins on its axis while circling the sun.}\\
      the earth's axis 地軸\\
      rotation 自轉; (如繞軸或中心的)旋轉\\
      revolution 公轉; (天體的)周轉\\
    \item
      orbit \textipa{/"OrbIt/} 軌道 (v. 依照軌道繞某天體運行)
    \item
      meteor \textipa{/"mitI\textrhookschwa/} 流星\\
      meteor shower 流星雨\\
      meteorite \textipa{/"mitI@r""aIt/} 隕石\\
      meteoroid \textipa{/"mitI@r""OId/} 流星體\\
    \item
      Hubble Space Telescope 哈伯太空望遠鏡
    \item
      satellite \textipa{/"s\ae{}t\r*l{}""aIt/} 衛星 \textcolor{red}{*}\marginpar{\scriptsize The Moon is a satellite of the Earth.}\\
      satellite station 太空站
    \item
      astronaut \textipa{/"\ae{}str@""nOt/} 太空人\\
      astronomer \textipa{/@"strAn@m\textrhookschwa/} 天文學家\\
      astronomical observatory \textipa{/""\ae{}str@"nAmik\r*l{} @b"z\textrhookrevepsilon{}v@""torI/} 天文台\\
      astronomical telescope 天文望遠鏡
    \item
      planet 行星\\
      secondary planet (行星的)衛星\\
      planetary \textipa{/"pl\ae{}n@""tErI/} 行星的\\
      planetarium \textipa{/""pl\ae{}n@"tErI@m/} 天文館
    \item
      the Big Bang theory 宇宙大爆炸生成論\footnote{稠密聚結的氣態物質的一次性宇宙爆炸,可能發生於 100 至 150 億年前,並構成宇宙的起源}\\
      the Big Dipper 北斗七星
    \item
      eclipse \textipa{/I"klIps/} 蝕\\
      \parbox[]{140pt}{a solar eclipse 日蝕\\total solar eclipse 日全蝕\\partial solar eclipse 日偏蝕}
      \parbox[]{140pt}{a lunar eclipse 月蝕\\total lunar eclipse 月全蝕\\partial lunar eclipse 月偏蝕}\\
      wax 漸盈(常指月亮)\\
      wane 漸虧\\
      wax and wane 盛衰消長 \textcolor{red}{*}\marginpar{\scriptsize The moon waxes till it becomes full, and then wanes.}\\
      crescent \textipa{/"krEs\d{n}t/} 新月\\
      half moon 半月(形)\\
      full moon 滿月\\
      syzygy \textipa{/"sIz@dZI/} 朔望\footnote{指日月地約略呈一直線 http://www.phy.cuhk.edu.hk/astroworld/dictionary/dictionary\_{}lunar.html}
    \item
      black hole 黑洞\footnote{重力極大的一塊空間,物質只進不出,就連光也無法逃逸,科學家認為黑洞由巨大的恆星崩潰而成}
    \item
      Polaris \textipa{/po"lErIs/} 北極星(= North Star = polestar \textipa{/"pol""stAr/})\\
      the polar lights; aurora \textipa{/O"ror@/} 極光\\
      aurora australis (= southern lights) 南極光\\
      aurora borealis (= northern light) 北極光
    \item
      vernal equinox 春分 (約在三月廿一日)(= vernal point)\\
      autumnal equinox \textipa{/O"t2mn\r*l{} "ikw@""nAks/} 秋分(約在九月廿二日)\\
      summer solstice \textipa{/"sAlstIs/} 夏至\\
      winter solstice 冬至
    \item
      configuration 輪廓; 形狀; 外型(如地形)
    \item
      asteroid \textipa{/"\ae{}strOId/} 小行星\\
      asteroid belt 小行星帶 (Celetial bodies between Mars and Jupiter)
    \item
      atmosphere \textipa{/"\ae{}tm@s""fIr/} 大氣層; 大氣
    \item
      comet 彗星\footnote{一種繞行太陽冰和塵埃的混合物,在太陽附近會形成離子尾、塵埃尾和明亮的慧髮}\\
      coma \textipa{/"kom@/} 彗髮\footnote{彗星周圍因太陽風而散佈的顆粒}\\
    \item
      cosmos \textipa{/"kAzm@s/} 宇宙\\
      chaos 混沌;天地未開的狀態
    \item
      gravity \textipa{/"gr\ae{}v@tI/} 重力;地心引力\\
      gravitational \textipa{/"gr\ae{}v@"teS@n\r*l{}/} (萬有引力的)\\
      zero gravity 無重力(狀態)
    \item
      light year 光年\footnote{天文學所用的距離單位,為光一年間所走的距離}\\
      leap year 閏年\footnote{以三百六十六日為一年的年份,二月份有二十九日}
    \item
      nuclear fusion 核融合\\
      neclear fission 核分裂
    \item
      aphelion \textipa{/\ae{}"filI@n/} 遠日點\\
      perihelion \textipa{/""pErI"hilI@n/} 近日點
    \item
      apogee \textipa{/"\ae{}p@""dZi/} 遠地點\\
      perigee \textipa{/"pEr@""dZi/} 近地點
    \item
      corona \textipa{/k@"ron@/} 光暈;日冕
    \item
      hydrogen \textipa{/"haIdr@dZ@n/} 氫
    \item
      nova 新星\\
      white dwarf 白矮星
  \end{enumerate}

  \chapter{Biology 生物學}
  生物學考題包含 zoology 動物學 entomology 昆蟲學 ornithology 鳥類學 marine biology 海洋生物學 microbiology 微生物學 animal
  behavior 動物行為學 botany 植物學,以動物學的昆蟲學、爬蟲、鳥類、和海洋生物學為主,植物學比例較低。

  \begin{enumerate}
    \item
      primate \textipa{/"praImIt/} 靈長目動物(包括人猴猿)\\
      reptile \textipa{/"rEptaIl/} 爬蟲類的冷血動物 \textcolor{red}{*}\marginpar{\scriptsize Whose body temperature changes according to the temperature around it, and that usually lays eggs to have babies.}\\
      mammal \textipa{/"m\ae{}m\r*l/} 哺乳動物\\
      rodent \textipa{/"rod\r*nt/} 囓齒類動物(鼠類和水獺)\\
      amphibian \textipa{/\ae{}m"fIbI2n/} 兩棲動物\\
      crustacean \textipa{/kr2s"teS@n/} 甲殼鋼動物(蟹和龍蝦)\\
      shellfish 貝類\\
      invertebrate 無脊椎動物\\
      mollusk \textipa{/"mAl@sk/} 軟體動物\\
      plankton \textipa{/"pl\ae{}Nkt@n/} 浮游生物\\
      marsupial \textipa{/mAr"supi@l/} 有袋動物\\
      ruminant \textipa{/"rum@n@nt/} 反芻動物\\
      poultry \textipa{/"poltrI/} 家禽\\
      arthropods \textipa{/"ArTr@""pAd/} 節肢動物(蜈蚣,昆蟲,蜘蛛)\\
      coral 珊瑚\\
      coral reef 珊瑚礁\\
      sponge 海綿動物\\
      fungus \textipa{/"f2Ng@s/} 菌類動物 \textcolor{red}{*}\marginpar{\scriptsize Mushrooms, toadstools and mildew are all fungi.\\蘑菇、蕈、白霉都是菌類植物}
    \item
      mutation (生物)突變、變異\\
    \item
      predator \textipa{/"prEd@t\textrhookschwa/} 掠奪者\\
      prey; quarry \textipa{/"kwOrI/} 被捕食者\\
      predatory 肉食的;捕食其他動物的\\
      predation \textipa{/prI"deS@n/} 掠奪;掠食
    \item
      ecology 生態學\\
      ecosystem \textipa{/"iko""sIst@m/} 生態系統\\
      biosphere \textipa{/"baI@"sfIr/} 生物圈\\
      ozone \textipa{/"oaon/} layer 臭氧層\\
      ozone depletion 臭氧的耗盡
    \item
      paleontology \textipa{/""pelIAn"tAl@dZI/} 古生物學\\
      fossil \textipa{/"fAs\r*l/} 化石\\
      pretrify 使石化 \textcolor{red}{*}\marginpar{\scriptsize Petrified wood/trees etc means that they have changed into stone over a long period of time}
    \item
      anatomy \textipa{/@"n\ae{}t@mI/} 解剖學\footnote{The art of science of dissection or of the structure of plants and animals, esp. of main}\\
      anatomical 解剖學\\
      clinical anatomy 臨床解剖學\\
      pathological \textipa{/""p\ae{}T@"lAdZIk@l/} anatomy 病理解剖學\\
      skeleton \textipa{/"skEl@t\r*n/} 骸骨
    \item
      species \textipa{/"spiSiZ/} (生物)種\\
      taxonomy \textipa{/t\ae{}ks"An@mI/} 分類學\footnote{動植物的分類學︰分為界 kindom、門 phylum \textipa{/"faIl@m/} (動物) division (植物)、綱 class、目 order、科 family、屬 genus \textipa{/"dZin@s/}、種 species }
    \item
      Charles Robert Darwin 達爾文 \footnote{演化論創始人,著有物種起源---物競天擇 The Origin of Species by Means of Natural Selection}\\
      evolve \textipa{/I"vAlv/} 進化 \marginpar{\scriptsize grey moth \& black moth prove natural selection---survival of the fittest.}\\
      lichen \textipa{/"laIk@n/} 地衣\marginpar{\scriptsize Lichens consist of fungi and algae}\\
      hereditary \textipa{/h@"rEd@""tErI/} traits 遺傳的特徵
    \item
      hibernate 冬眠 (n.$\to$ hibernation)
    \item
      herd 成群結隊\marginpar{\scriptsize a herd of elephants/cattle}\\
      flock 較小的家畜或鳥類之同食同飛者\marginpar{\scriptsize a flock of birds}\\
      pack 指聚群獵食的野獸\footnote{suggest animals hunting in company}\marginpar{\scriptsize a pack of wolves}\\
      school 指數目繁多的魚\marginpar{\scriptsize a school of whaels}\\
    \item
      tentacle [1]觸手;觸鬚 [2]植物葉上的觸毛
    \item
      antenna \textipa{/\ae{}n"tEn@/} (昆蟲的)觸角
    \item
      mandible \textipa{/"m\ae{}nd@b\d{l}/} [1]下頜骨 [2]鳥喙 [3] (節族動物 arthropods 的)大顎(用來咬碎、吸吮或刺)
    \item
      beak \textipa{/bik/} 鳥嘴\\
      fang 食肉動物的牙;(蛇的)毒牙\\
      plume \textipa{/plum/} 大羽毛\footnote{a large feather}\\
      plumage \textipa{/"plumidZ/} (鳥的)全身羽毛\\
      claw \textipa{/klO/} (動物的)爪;蝦蟹的鉗螯\\
      paw \textipa{/pO/} 腳掌
    \item
      camouflage \textipa{/"k\ae{}mu""flAZ/} 掩飾;偽裝 \textcolor{red}{*}\marginpar{\scriptsize Soldiers learn camouflage technique.\\[2em] Soldiers had camouflaged the trucks with branches and dirt.}\\
      chameleon \textipa{/k@"milj@n/} 變色龍
    \item
      venom \textipa{/"vEn@m/} (動物或昆蟲的)毒液
    \item
      pollen \textipa{/"pAl@n/} 花粉\\
      petal \textipa{/"pEt\d{l}/} 花瓣\\
      nectar \textipa{/"nEkt\textrhookschwa/} 花蜜 \textcolor{red}{*}\marginpar{\scriptsize The bees spend the summer collecting nectar and turning it into honey.\\蜜蜂夏天採集花蜜並釀成蜂蜜}
    \item
      tissue (植物或動物的)組織
    \item
      weed \textipa{/wid/} [1]雜草 [2] 海草\\
      seaweed \textipa{/"\ae{}lg@/} 海藻
    \item
      beehive \textipa{/"bi""haIv/} 蜂窩
    \item
      habitat \textipa{/"h\ae{}b@""t\ae{}t/} (動植物的)產地;棲息地\\
      rookery \textipa{/"ruk@rI/} 鳥群或獸群\\
      seal rookery 海豹繁殖的地方\\
      roost \textipa{/rust/} 鳥棲息處
    \item
      hatch \textipa{/h\ae{}tS/} 孵化\\
      incubate \textipa{/"Inkju""bet/} 孵卵\\
      spawn \textipa{/"spOn/} 產卵
    \item
      fertilize [1]授精(= to impregnate) [2]施肥(= to enrich with fertilizer)
    \item
      airborne \textipa{/"Erborn/} 經由空氣傳送的
    \item
      food chain 食物鏈
    \item
      carnivorous \textipa{/kAr"nIv@r@s/} 肉食的\\
      carnivore \textipa{/"kArn@""vOr/} 肉食動物\\
      herbivorous \textipa{/h\textrhookrevepsilon{}"bIv@r@s/} 草食的\\
      herbivore \textipa{/"h\textrhookrevepsilon{}b@""vOr/} 草食動物\\
      omnivorous \textipa{/Am"nIv@r@s/} 雜食\\
      omnivore \textipa{/"Amn@""vOr/} 雜食動物
    \item
      nocturnal \textipa{/nAk"t\textrhookrevepsilon{}n\d{l}/} 夜間活動的\\
      diurnal \textipa{/daI"\textrhookrevepsilon{}n\d{l}/} 日間活動的
    \item
      reproduce; regenerate 再生;複製 \textcolor{red}{*}\marginpar{\scriptsize Lobsters are able to reproduce claws when they are torn off\\龍蝦的鉗被砍掉後能再生出來}
    \item
      extant \textipa{/"Ekst@nt/} 現存的\\
      extinct \textipa{/Ik"stINkt/} 絕種的;不存在的(n.$\to$ extinction=extermination)\\
      endangered \textipa{/In"dendZ\textrhookschwa{}d/} 面臨絕種危機的
    \item
      parasite \textipa{/"p\ae{}r@""saIt/} 寄生動植物 \textcolor{red}{*}\marginpar{\scriptsize 如蝨 lice\\水蛭 leech} $\leftrightarrow$ host \textipa{/host/} 宿主\\
      parasitic \textipa{/""p\ae{}r@"sItIk/} (= parasitical)
    \item
      scale \textipa{/skel/} 鱗\\
      gill \textipa{/gIl/} 鰓\\
      fin \textipa{/fIn/} 鰭 \textcolor{red}{*}\marginpar{\scriptsize dorsal fin 背鰭\\ventral fin 腹鰭}\\
      filppter \textipa{/"flIp\textrhookschwa/} 鰭狀肢\footnote{A broad fin, arm or paddle used in swimming, as that of the whale, seal, or turtle}
    \item
      gene \textipa{/dZin/} 基因\\
      genetic \textipa{/dZ@"nEtIk/} 基因的;遺傳的\\
      genetic defects 先天缺陷\\
      genetic mutations 基因突變
    \item
      chromosome \textipa{/"krom@""som/} 染色體\\
    \item
      cactus \textipa{/"k\ae{}kt@s/} 仙人掌\\
      stem \textipa{/stEm/} 莖\\
      starch \textipa{/stArtS/} 澱粉\\
      bark 樹皮\\
      resin \textipa{/"rEz@n/} 樹脂\\
      arid \textipa{/"\ae{}rId/} 乾旱的;不毛的\\
      aridity \textipa{/\ae{}"rId@tI/} 乾燥\\
      chlorophy \textipa{/"klor@fIl/} 葉綠素\\
      photosynthesis \textipa{/""fot@"sInT@sIs/} 光合作用
    \item
      metabolism \textipa{/m@"t\ae{}b\d{l}""Iz@m/}\\
    \item
      anterior 前面的\\
      posterior \textipa{/"postIri\textrhookschwa/} 後面的\\
      ventral \textipa{/"vEntr@l/} 腹部的\\
      dorsal \textipa{/"dOrs\d{l}/} 背部的
    \item
      pest \textipa{/pEst/} 害蟲\\
      pesticide \textipa{/"pEstI""saId/} 殺蟲劑\\
      herbicide \textipa{/"h\textrhookrevepsilon{}b@""saId/} 除草劑
  \end{enumerate}

  \chapter{Anthropology \& Archaeology 考古與史前人類發展}

  \chapter{Chemistry 化學}

  \chapter{History 歷史}

  \chapter{Arts \& Humanities 藝術和人文}

  \chapter{Architecture 建築學}

  \chapter{Geology \& Earth Science 地理與地球科學}

  \chapter{Physiology and Medicine 生理學和醫學}

\end{document}
