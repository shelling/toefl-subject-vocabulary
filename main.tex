\documentclass[twoside,b5paper]{book}

\usepackage{color}
\usepackage{url}
\usepackage{fontspec}
\usepackage{graphicx}
\usepackage{textcomp}
\usepackage{tipa}

\setmainfont{BiauKai}
\newfontfamily\kai{BiauKai}
\newfontfamily\arial{Arial}

\evensidemargin=0pt
\oddsidemargin=0pt

\parindent=0pt
\marginparwidth=60pt
\topmargin=0pt
\headheight=0pt
%% \headsep=0pt

\voffset=0pt
\hoffset=0pt

\XeTeXlinebreaklocale "zh"

\begin{document}
  \title{TOEFL Subject Vocabulary\\ 托福學科單字}
  \author{馬翎 編繹 shelling 整理}
  \maketitle

  \tableofcontents % uncomment this line to make toc

  \chapter{Astronomy \kai 天文學}
  \begin{enumerate}
    \item
      galaxy \textipa{/"g\ae{}l@ksI/} 銀河系\\
      elliptical galaxy 橢圓的星系\\
      spiral galaxy 螺旋星系
    \item
      nebula \textipa{/"nEbj@l@/} 星雲\\
      spiral nebula 漩渦星座\\
      nebular 星雲的\\
      constellation \textipa{/""kAnst@"leS@n/} 星座 \textcolor{red}{*}\marginpar{\scriptsize a group of stars visible from Earth that forms a distinctive pattern and has a name}\\
      star cluster 星團
    \item
      radiation 輻射線; 放射 (v. radiate)
    \item
      magnetic field 磁場\\
      magnetic storm 磁暴\\
      magnetic pole 磁極\\
      magnetosphere (圍繞地球的)磁力層 \textcolor{red}{*}\marginpar{\scriptsize 地球大氣層外的外圈厚數千英里受地球磁場控制}\\
      magnitude (星的)光度
    \item
      particles; dust 微塵; 微粒
    \item
      interstellar \textipa{/""Int\textrhookschwa"stEl\textrhookschwa/} 位於各星之間的; 星際的\\
      stellar \textipa{/"stEl\textrhookschwa/} 星的\\
      stellar luminosity \textipa{/""lum@"nAs@tI/} 星的光度\\
      luminous \textipa{/"lum@n@s/} 發光的
    \item
      the solar system 太陽系\\
      solar spots / sunspots 太陽黑點\\
      Mercury 水星\\
      Venus 金星\\
      Earth 地球\\
      Mars 火星\\
      Jupiter 木星\textcolor{red}{*}\marginpar{\scriptsize In the solar system, the largest planet.}\\
      Saturn \textipa{/"s\ae{}t\textrhookschwa{}n/} 土星\\
      Uranus \textipa{/ju"ren@s/} 天王星\\
      Neptune \textipa{/""nEptjun/} 海王星\\
      Pluto \textipa{/"pluto/} 冥王星
    \item
      sphere \textipa{/sfIr/} 球狀體\\
      hemisphere \textipa{/"hEm@s""fIr/} 半球 \textcolor{red}{*}\marginpar{\scriptsize the Northern / Southern / Eastern / Wesern Hemisphere 北/南/東/西半球}\\
    \item
      celestial \textipa{/s@"lEstS@l/} 天上的\\
      celestial body 天體
    \item
      axis \textipa{/"\ae{}ksIs/} 軸\\
      the earth's axis 地軸\\
      \framebox{The Earth revolves around the sun.}\\
      \framebox{The Earth spins on its axis while circling the sun.}\\
      rotation 自轉; (如繞軸或中心的)旋轉\\
      revolution 公轉; (天體的)周轉\\
    \item
      orbit \textipa{/"OrbIt/} 軌道 (v. 依照軌道繞某天體運行)
    \item
      meteor \textipa{/"mitI\textrhookschwa/} 流星\\
      meteor shower 流星雨\\
      meteorite \textipa{/"mitI@r""aIt/} 隕石\\
      meteoroid \textipa{/"mitI@r""OId/} 流星體\\
  \end{enumerate}


  \chapter{Biology 生物學}

  \chapter{Anthropology \& Archaeology 考古與史前人類發展}

  \chapter{Chemistry 化學}

  \chapter{History 歷史}

  \chapter{Arts \& Humanities 藝術和人文}

  \chapter{Architecture 建築學}

  \chapter{Geology \& Earth Science 地理與地球科學}

  \chapter{Physiology and Medicine 生理學和醫學}

\end{document}
